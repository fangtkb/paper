% !TEX TS-program = pdflatex
% !TEX encoding = UTF-8 Unicode

% This is a simple template for a LaTeX document using the "article" class.
% See "book", "report", "letter" for other types of document.

\documentclass[11pt]{book} % use larger type; default would be 10pt

\usepackage[utf8]{inputenc} % set input encoding (not needed with XeLaTeX)

%%% Examples of Article customizations
% These packages are optional, depending whether you want the features they provide.
% See the LaTeX Companion or other references for full information.

%%% PAGE DIMENSIONS
\usepackage{geometry} % to change the page dimensions
\geometry{a4paper} % or letterpaper (US) or a5paper or....
% \geometry{margins=2in} % for example, change the margins to 2 inches all round
% \geometry{landscape} % set up the page for landscape
%   read geometry.pdf for detailed page layout information

\usepackage{graphicx} % support the \includegraphics command and options

% \usepackage[parfill]{parskip} % Activate to begin paragraphs with an empty line rather than an indent

%%% PACKAGES
\usepackage{booktabs} % for much better looking tables
\usepackage{array} % for better arrays (eg matrices) in maths
\usepackage{paralist} % very flexible & customisable lists (eg. enumerate/itemize, etc.)
\usepackage{verbatim} % adds environment for commenting out blocks of text & for better verbatim
\usepackage{subfig} % make it possible to include more than one captioned figure/table in a single float
% These packages are all incorporated in the memoir class to one degree or another...
\usepackage{amsmath,amssymb}

%%% HEADERS & FOOTERS
\usepackage{fancyhdr} % This should be set AFTER setting up the page geometry
\pagestyle{fancy} % options: empty , plain , fancy
\renewcommand{\headrulewidth}{0pt} % customise the layout...
\lhead{}\chead{}\rhead{}
\lfoot{}\cfoot{\thepage}\rfoot{}

%%% SECTION TITLE APPEARANCE
\usepackage{sectsty}
\allsectionsfont{\sffamily\mdseries\upshape} % (See the fntguide.pdf for font help)
% (This matches ConTeXt defaults)

%%% ToC (table of contents) APPEARANCE
\usepackage[nottoc,notlof,notlot]{tocbibind} % Put the bibliography in the ToC
\usepackage[titles,subfigure]{tocloft} % Alter the style of the Table of Contents
\renewcommand{\cftsecfont}{\rmfamily\mdseries\upshape}
\renewcommand{\cftsecpagefont}{\rmfamily\mdseries\upshape} % No bold!

%%% END Article customizations

%%% The "real" document content comes below...

\title{Thesis}
\author{Fang Ni}
%\date{} % Activate to display a given date or no date (if empty),
         % otherwise the current date is printed 

\begin{document}
\maketitle

\tableofcontents

\chapter{Introduction}
\section{First section}

Your text goes here.

\subsection{A subsection}

More text.

\chapter{Pairing model}

Pairing correlation plays important role near the Fermi energy. To examine the pairing dynamics influencing the structure of nuclei, we can only concentrate the nucleons near the Fermi energy. The Hamiltonian of pairing model is
\begin{align}
	H &= \sum_l \epsilon_l n_l - g \sum_{l,l'} S_l^+ S_{l'}^- ,
\end{align}
where
\begin{align}
	n_l &= \sum_m a^{\dag}_{lm}a_{lm} \\
        S_l^{+} &= \sum_{m>0}a_{lm}^{\dag}a_{l\overline{m}}^{\dag} ,
\quad   S_l^{-} = S_l^{+\dag} 
\end{align}

\section{Exact solution}

\section{TDHFB dynamics}


\chapter{Requantization of TDHFB in integrable system}

\section{Canonical quantization}

\section{Fourier decomposition}

\section{Stationary phase to the path integral}

\section{Result}

\chapter{Requantization of TDHFB in non-integrable system}

\section{Derivation of the collective subspace in adiabatic self-consistent collective coordinate method}

\section{Application of SPA in non-integrable system}

\section{Result}

\chapter{Discussion}

\chapter{Conclusion}

\end{document}
